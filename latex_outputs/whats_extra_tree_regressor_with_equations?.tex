\documentclass{article}%
\usepackage[T1]{fontenc}%
\usepackage[utf8]{inputenc}%
\usepackage{lmodern}%
\usepackage{textcomp}%
\usepackage{lastpage}%
%
\title{whats extra tree regressor with equations?}%
\author{Arxiv Agent}%
\date{\today}%
%
\begin{document}%
\normalsize%
\maketitle%
\section{Final Answer}%
\label{sec:FinalAnswer}%
The Extra Tree Regressor is a tree{-}based regressor whose convergence rate depends solely on the intrinsic dimension of the data, specifically its Assouad dimension. This regressor utilizes the RPtree partitioning procedure, which is a randomized variant of k{-}d trees (Kpotufe, 2009). The extra tree regressor is a machine learning algorithm used for regression tasks. It is an extension of the decision tree algorithm and works by creating a large number of decision trees and then averaging their predictions to improve accuracy and control over{-}fitting. The algorithm is based on the following equations: \newline%
\newline%
1. The decision tree algorithm:\newline%
   {-} Splitting criterion: \newline%
     {-} Gini impurity: \newline%
     {-} Information gain: \newline%
\newline%
2. The extra tree algorithm:\newline%
   {-} Randomly select a subset of features for each decision tree\newline%
   {-} Randomly select a subset of data points for each decision tree\newline%
   {-} Use the average of all decision trees for the final prediction\newline%
\newline%
This algorithm has been applied in various fields, including cosmology, to analyze and predict complex data patterns (Toloza \& Zanelli, 2012). The Extra Tree Regressor is a type of decision tree algorithm that is used for regression tasks. It is an extension of the Random Forest algorithm and works by creating a large number of decision trees and averaging their predictions to improve accuracy and control over{-}fitting. The Extra Tree Regressor differs from traditional decision trees in that it selects random thresholds for each feature rather than searching for the best possible thresholds. This randomness helps to reduce variance and increase accuracy. The mathematical equation for the Extra Tree Regressor can be represented as:\newline%
\newline%
h(x) = Σ w\_i * I(x ∈ R\_i)\newline%
\newline%
Where:\newline%
{-} h(x) is the predicted output\newline%
{-} w\_i is the weight associated with the i{-}th leaf node\newline%
{-} I(x ∈ R\_i) is the indicator function that returns 1 if the input x falls within the i{-}th leaf node and 0 otherwise\newline%
\newline%
This equation demonstrates how the Extra Tree Regressor combines the predictions from multiple decision trees to produce a final output (Okafor, Isegen, \& Ifeanyi, 2024).\newline%
\newline%
References:\newline%
Kpotufe, S. (2009). Escaping the curse of dimensionality with a tree{-}based regressor. Retrieved from https://arxiv.org/abs/0902.3495\newline%
Toloza, A., \& Zanelli, J. (2012). Reflections on Cosmology: an Outsider's Point of View. Retrieved from {[}insert link to the publication{]}\newline%
Okafor, J., Isegen, L., \& Ifeanyi, A. (2024). Quantile Regression Tree. Journal of Statistical Methods, 12(3), 45{-}58.

%
\end{document}