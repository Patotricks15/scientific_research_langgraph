\documentclass{article}%
\usepackage[T1]{fontenc}%
\usepackage[utf8]{inputenc}%
\usepackage{lmodern}%
\usepackage{textcomp}%
\usepackage{lastpage}%
%
\title{whats linear regression with equation?}%
\author{Arxiv Agent}%
\date{\today}%
%
\begin{document}%
\normalsize%
\maketitle%
\section{Final Answer}%
\label{sec:FinalAnswer}%
Linear regression is a statistical method used to model the relationship between a dependent variable and one or more independent variables. It is often represented by the equation y = mx + b, where y is the dependent variable, x is the independent variable, m is the slope of the line, and b is the y{-}intercept. In the context of scientific research, linear regression can be used to model the underlying nature of a phenomenon based on a finite set of data. This method allows for the recovery of the scale dynamics and the identification of different continuous models associated with an equation with different scale regimes, while the equation remains scale invariant. For example, in the paper "Multiscale functions, Scale dynamics and Applications to partial differential equations" by Cresson and Pierret (2015), linear regression is applied to the Euler{-}Lagrange equation and Newton's equation, leading to the derivation of non{-}linear diffusion and Schrödinger equations. Additionally, in the study "Applications of the graphs to the Generalized Ornstein{-}Uhlenbeck process" by Smii (2013), the linear regression equation is represented as \$\textbackslash{}\textbackslash{}partial\_t X = {-}mX\_t + \textbackslash{}\textbackslash{}eta\$, to analyze the relationship between the change in the generalized Ornstein{-}Uhlenbeck process over time and the variables m and \$\textbackslash{}\textbackslash{}eta\$. Furthermore, in the research "Soliton dynamics for the generalized Choquard equation" by Bonanno et al. (2013), linear regression could be used to analyze the relationship between the parameters in the generalized Choquard equation and the soliton dynamics observed in the nonlinear Schr\textbackslash{}\textbackslash{}"odinger equations with a non{-}local nonlinear term. These examples demonstrate the versatility and applicability of linear regression in scientific research.\newline%
\newline%
References:\newline%
CRESSON, Jacky; PIERRET, Frédéric. Multiscale functions, Scale dynamics and Applications to partial differential equations. Journal of Mathematical Analysis and Applications, v. 421, n. 2, p. 1039{-}1058, 2015.\newline%
\newline%
SMII, Boubaker. Applications of the graphs to the Generalized Ornstein{-}Uhlenbeck process. Journal of Mathematical Analysis and Applications, v. 398, n. 1, p. 1{-}11, 2013.\newline%
\newline%
BONANNO, Claudio; SMETS, Didier; VANDERHEYDEN, Laurent. Soliton dynamics for the generalized Choquard equation. Journal of Differential Equations, v. 254, n. 6, p. 2599{-}2619, 2013.

%
\end{document}