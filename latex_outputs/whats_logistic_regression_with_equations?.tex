\documentclass{article}%
\usepackage[T1]{fontenc}%
\usepackage[utf8]{inputenc}%
\usepackage{lmodern}%
\usepackage{textcomp}%
\usepackage{lastpage}%
%
\title{whats logistic regression with equations?}%
\author{Arxiv Agent}%
\date{\today}%
%
\begin{document}%
\normalsize%
\maketitle%
\section{Final Answer}%
\label{sec:FinalAnswer}%
Logistic regression is a statistical method used for predicting the probability of a binary outcome based on one or more predictor variables. It uses the logistic function to model the relationship between the dependent variable and the independent variables. The equation for logistic regression is:\newline%
\newline%
P(Y=1|X) = 1 / (1 + e\^{}{-}(b0 + b1X1 + b2X2 + ... + bkXk))\newline%
\newline%
Where P(Y=1|X) is the probability of the dependent variable being 1 given the independent variables X, e is the base of the natural logarithm, and b0, b1, b2, ..., bk are the coefficients of the independent variables. Logistic regression is a type of classifier that makes predictions based on the relationship between the independent variables and the dependent variable. It is the "soft" variant of perceptron learning, which is a linear classifier, and uses gradient ascent to find the optimal coefficients for the logistic regression model.

%
\end{document}