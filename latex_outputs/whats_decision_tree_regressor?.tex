\documentclass{article}%
\usepackage[T1]{fontenc}%
\usepackage[utf8]{inputenc}%
\usepackage{lmodern}%
\usepackage{textcomp}%
\usepackage{lastpage}%
%
\title{whats decision tree regressor?}%
\author{Arxiv Agent}%
\date{\today}%
%
\begin{document}%
\normalsize%
\maketitle%
\section{Final Answer}%
\label{sec:FinalAnswer}%
The decision tree regressor is a machine learning algorithm used for predictive modeling and regression analysis. It is a type of decision tree that is used to predict continuous values rather than discrete classes. Decision tree regressors work by recursively partitioning the input space into regions and fitting a simple model (e.g., a constant value) within each region. In the context of Explainable AI, decision trees are known for their inherent explainability and can be used as surrogate models for complex black box AI models or as approximations of parts of such models (Ozaki et al., 2024). The study by Ozaki et al. (2024) demonstrates the application of decision tree regressors in addressing the interpretability and trustworthiness of AI models, particularly in the context of identifying biases and ensuring fairness. Additionally, the decision tree regressor is used as part of an algorithm for updating a decision tree while minimizing the number of changes to the tree that a human would need to audit, as proposed in the research article by Simmons et al. (2024). This algorithm aims to achieve a balance between final accuracy and the number of changes to audit, highlighting the versatility and practical applications of decision tree regressors in various scientific domains.

%
\end{document}