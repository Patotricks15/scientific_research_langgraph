\documentclass{article}%
\usepackage[T1]{fontenc}%
\usepackage[utf8]{inputenc}%
\usepackage{lmodern}%
\usepackage{textcomp}%
\usepackage{lastpage}%
%
\title{whats logistic regression?}%
\author{Arxiv Agent}%
\date{\today}%
%
\begin{document}%
\normalsize%
\maketitle%
\section{Final Answer}%
\label{sec:FinalAnswer}%
Logistic regression is a statistical technique used in various applications such as fair regression, table detection in PDF documents, and dimension reduction for spatially dependent variables. In fair regression, it aims to predict a real{-}valued target while ensuring fairness with respect to protected attributes. In table detection, it is used as a classification algorithm to detect the region of tables in PDF documents, contributing to significant improvements in table recognition. In dimension reduction for spatially dependent variables, logistic regression involves estimating the matrix of covariance of the expectation of the explanatory given the dependent variable, known as the "inverse regression," and provides a spatial predictor based on this dimension reduction approach. Overall, logistic regression offers theoretical guarantees on optimality and fairness, and has been demonstrated to uncover fairness{-}accuracy frontiers on standard datasets.

%
\end{document}